\documentclass[9pt,twocolumn]{extarticle}

\usepackage[hmargin=0.5in,tmargin=0.5in]{geometry}
\usepackage{amsmath,amssymb}
\usepackage{times}
\usepackage{graphicx}
\usepackage{subfigure}

\usepackage{cleveref}
\usepackage{color}
\newcommand{\TODO}[1]{\textcolor{red}{#1}}

\newcommand{\FPP}[2]{\frac{\partial #1}{\partial #2}}
\newcommand{\argmin}{\operatornamewithlimits{arg\ min}}
\author{Siwang Li}

\title{MPRGP extension}

%% document begin here
\begin{document}
\maketitle

\setlength{\parskip}{0.5ex}

\section{Step Limit}
Given a point $x\in R^{3n}$ and a direction $d\in R^{3n}$, we compute 
\begin{equation} 
  \alpha = \max_{\alpha}\{\alpha\ge 0: x+\alpha d\in \Omega_B\}
\end{equation}
For each $x_i,d_i\in R^3$, we compute
\begin{equation}
  \alpha_i=\left\{ \begin{array}{rl}
      0, & d_i\cdot n_j > 0\\
      \alpha_i:x_i+\alpha_id \in P_j, & d_i\cdot n_j <= 0
    \end{array} \right.
\end{equation}
for each plane $P_j$ with norm of $n_j$. Finally, we compute
\begin{equation} 
  \alpha = \min \alpha_i
\end{equation}

Suppose the plane is defined as $n\cdot y + p = 0$, then we need to solve 
\begin{equation}
  n\cdot(x-\alpha d)+p = 0
\end{equation}
for $\alpha$, i.e
\begin{equation}
  \alpha = \frac{n\cdot x+p}{n\cdot d}
\end{equation}

\section{Project $P_{\Omega_B}$}
Given an point $x_i\in R^3$, we project it to the feasible region $y_i=P_{\Omega_B}(x_i)$ by solving
\begin{equation} 
  y_i = \min_{y} \frac{1}{2}\|y-x_i\|_2^2 \mbox{, s.t. }y \cdot n_j + d_j \ge 0 \mbox{ for each plane }(n_j,d_j).
\end{equation}

\section{Compute $\phi$}
Given the gradient $g_i(x_i)\in R^3$, then
\begin{equation}
  \phi_i = \left\{ \begin{array}{rl}
     g_i & p_N = 0\\
     g_i \mbox{ project on plane} & p_N = 1\\
     g_i \mbox{ project on line}& p_N = 2\\
     0 & p_N \ge 3\\
    \end{array} \right.
\end{equation}
where $p_N$ is the number of planes that $x_i$ belong to.

\section{Compute $\beta$}
Given the gradient $g_i(x_i)\in R^3$, then
\begin{equation} 
  \beta_i = \min_{\beta} \frac{1}{2}\|(-g_i)-(-\beta)\|_2^2 \mbox{, s.t. } (-\beta) \cdot n_j \ge 0 \mbox{ for } j\in P_i \mbox{, and } \beta \cdot \phi_i=0.
\end{equation}
Let $y = -\beta$, then
\begin{equation} 
  y = \min_{y} \frac{1}{2}\|y-(-g_i)\|_2^2 \mbox{, s.t. } y \cdot n_j \ge 0 \mbox{ for } j\in P_i \mbox{, and } y \cdot \phi_i=0.
\end{equation}
and $\beta = -y$. Here $P_i$ is the planes that contains $x_i$. If $|P_i|=0$, then $\beta_i=0$.

\section{Decide Face}


\section{Compute $\tilde{\phi}$}
According to the equation (5.64) in the book \cite{book}, we have
\begin{equation}
  \tilde{\phi}(x_j,{\alpha}) = \frac{1}{\alpha}x_i-\beta_i(x_i)-\frac{1}{\alpha} P_{\Omega_B}(x_i-\alpha g(x_i))
\end{equation}

\section{Diagonal Precondition}
For each $x_i\in R^{3}$, if $|P_i| > 0$, then the three corresponding $3$ elements in the Diagonal Precondition Matrix $M$ is 1. Otherwise it is equal to the inverse of the diagonal elements of the input SPD matrix $A$.

% references
\begin{thebibliography}{99}
\bibitem{book} Optimal Quadratic Programming Algorithms With Applications to Variational Inequalities.
\end{thebibliography}

\end{document}
